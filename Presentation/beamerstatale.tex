\documentclass{beamer}
\usepackage{amsfonts,amsmath,oldgerm,graphicx,epstopdf}
\usepackage{booktabs}
\epstopdfsetup{outdir=./}
\setbeamertemplate{caption}{\color{maincolor}Fig. \color{darkgray} \raggedright\insertcaption\par}

\usetheme{sintef}

% \newcommand{\testcolor}[1]{\colorbox{#1}{\textcolor{#1}{test}}~\texttt{#1}}

\usefonttheme[onlymath]{serif}

\titlebackground*{assets/background}

\newcommand{\hrefcol}[2]{\textcolor{cyan}{\href{#1}{#2}}}

\title{Use of a ligand containing a pyrazole functionality and two carboxylate groups to construct new MOFs for applications in gas storage}
\subtitle{Using \LaTeX\ to prepare slides}
\course{Master's Degree in Computer Science}
\author{\href{mailto:ciao@gio.im}{Stefano Marton}}
\IDnumber{962848}

\begin{document}
\maketitle

\begin{frame}

	This template is a based on \hrefcol{https://www.overleaf.com/latex/templates/sintef-presentation/jhbhdffczpnx}{SINTEF Presentation} from \hrefcol{mailto:federico.zenith@sintef.no}{Federico Zenith} and its derivation \hrefcol{https://github.com/TOB-KNPOB/Beamer-LaTeX-Themes}{Beamer-LaTeX-Themes} from Liu Qilong

	\vspace{\baselineskip}

	In the following you find a brief introduction on how to use \LaTeX\ and the beamer package to prepare slides, based on the one written by \hrefcol{mailto:federico.zenith@sintef.no}{Federico Zenith} for \hrefcol{https://www.overleaf.com/latex/templates/sintef-presentation/jhbhdffczpnx}{SINTEF Presentation}

	% This template is released under \hrefcol{https://creativecommons.org/licenses/by-nc/4.0/legalcode}{Creative Commons CC BY 4.0} license

\end{frame}

\backmatter
\end{document}
