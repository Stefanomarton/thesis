%++++++++++++++++++++++++++++++++++++++++ 
\documentclass[a4,11pt,openright]{report}
\usepackage{Preamble}
%++++++++++++++++++++++++++++++++++++++++
\title{Bachelor Thesis}
\author{Stefano Marton}
\date{2022/2023}
\makeatletter
\def\cleardoublepage{\clearpage%
	\if@twoside
		\ifodd\c@page\else
			\vspace*{\fill}
			\hfill
			\begin{center}
				This page intentionally left blank.
			\end{center}
			\vspace{\fill}
			\thispagestyle{empty}
			\newpage
			\if@twocolumn\hbox{}\newpage\fi
		\fi
	\fi
}
\makeatother

\begin{document}

\pagenumbering{Roman} %cool numbering for early pages
\begin{titlepage}
	\begin{figure}[h]
		\centering
		\includegraphics[width=6cm]{Images/Logo_Università_degli_Studi_di_Milano.pdf}
	\end{figure}
	\centering \Large
	\textbf{UNIVERSITY OF MILAN}\\
	\textbf{FACULTY OF CHEMICAL SCIENCE AND TECHNOLOGY}

	BACHELOR'S DEGREE IN CHEMISTRY
	\begin{center}
		\vspace*{2cm}
		\LARGE
		\textbf{Investigation of the synthesis of a new pyrazole dicarboxylate ligand for the construction of new MOFs}
	\end{center}
	\vspace*{2.6cm}
	\begin{flushleft}
		\fontsize{12.5}\selectfont \textbf{\fontsize{12.5}\selectfont Supervisor}: Professor Lucia Carlucci\\
		\fontsize{12.5}\selectfont \textbf{\fontsize{12.5}\selectfont Co-supervisor}: Professor Pierluigi Mercandelli \\
	\end{flushleft}
	\vspace*{1.3cm}
	\begin{flushright}
		\begin{tabular}{l}
			\fontsize{12.5}\selectfont \textbf{\fontsize{12.5}\selectfont Candidate}: Stefano Marton \\
			\fontsize{12.5}\selectfont \textbf{\fontsize{12.5}\selectfont ID number}: 962848
		\end{tabular}
	\end{flushright}
	\vspace{2cm}
	\begin{center}
		\fontsize{12.5}\selectfont Academic Year 2022-2023
	\end{center}
	\vspace*{\fill}
\end{titlepage}

\afterpage{\blankpage}
\cleardoublepage

\chapter*{} %hacky
\vspace*{\fill}
\begin{flushleft}
	\topskip0pt
	\it
	First and foremost, I want to extend my deepest appreciation to my loving parents. Your endless support, belief in me, and sacrifices have been the driving force behind my pursuits. Without you, this achievement would not have been possible.

	To my dear friends, your encouragement and understanding during this challenging period have been a source of strength. Your support meant the world to me, and I am truly grateful for the camaraderie we share.

	Special thanks go to my friend Stefano, who not only provided invaluable input and inspiration for my work but also made sure we had quality downtime together. Your friendship has been a cherished aspect of this journey.

	I would like to express my gratitude to Giulia, who provided me with immense support and patience. Thank you for guiding me on an easier path to follow.

	I extend my sincere thanks to professor Carlucci for your guidance, mentorship, and valuable insights throughout my research. Delia, I appreciate your assistance and guidance as well. To all my fellow group members, your collaboration and the incredible lab environment we had were instrumental in my success.

	Also, I feel I should thank and remember my high school teacher Cristina's inspiration to look a little further than the garden outside the house.

	I am deeply grateful to everyone who has contributed to my academic and personal growth. This work would not have been possible without your support and encouragement, you are all in my heart.
	\vspace*{\fill}

\end{flushleft}
\vspace*{\fill}

\afterpage{\blankpage}

\addcontentsline{toc}{chapter}{Abbreviations}
\chapter*{Abbreviations}
\begin{acronym}
	\acro{DikDiEst}{\iupac{\small Dimethyl 4,4'-malonyldibenzoate}}
	\acro{DikDiAc}{\iupac{\small 4,4'-malonyldibenzoic acid}}
	\acro{PyrDiEst}{\iupac{\small Methyl 4,4'-(1H-pyrazole-3,5-diyl)dibenzoate}}
	\acro{PyrDiAc}{\iupac{\small Dimethyl 4,4'-malonyldibenzoate}}
	\acro{UnsDiEst}{\iupac{\small Methyl 4-[(1E)-3-[4-(methoxycarbonyl)phenyl]-3-oxoprop-1-en-1-yl]benzoate}}
	\acro{DikDiCN}{\iupac{\small Dimethyl 4,4'-malonyldibenzoate}}
	\acro{DikCN}{\iupac{\small 4‐[(2Z)‐3‐hydroxy‐3‐phenylprop‐2‐enoyl]benzonitrile}}
	\acro{DikBiCN}{\iupac{\small 4'‐[(2Z)‐3‐{4'‐cyano‐[1,1'‐biphenyl]‐4‐yl}‐3‐hydroxyprop‐2‐enoyl]‐[1,1'‐biphenyl]‐4‐carbonitrile}}
	\acro{PyrBiCN}{\iupac{\small 4'‐(5‐{4'‐cyano‐[1,1'‐biphenyl]‐4‐yl}‐1H‐pyrazol‐3‐yl)‐[1,1'‐biphenyl]‐4‐carbonitrile}}
	\acro{Fe III metalloligand}{\small \iupac{Tris[1,3-bis(4’-cyanophenyl)-1,3-propanedionato]iron(III)}}
	\acro{Co III metalloligand}{\small \iupac{Tris[1,3-bis(4’-cyanophenyl)-1,3-propanedionato]cobalt(III)}}
	\acro{Zn II metalloligand}{\small \iupac{Tetraethylammonium tris[1,3-bis(4’-cyanophenyl)-1,3propanedionato]cobaltate(II)}}
	\acro{Co II metalloligand}{\small \iupac{Tetraethylammonium tris[1,3-bis(4’-cyanophenyl)-1,3-propanedionato]zincate(II)}}
	\acro{Fe III MOF}{\iupac{\small \(\{[Fe^{III}(DikDiCN)_{3}Ag_{3}](ClO_{4})_{3}\}\cdot Solv\)}}
\end{acronym}

\afterpage{\blankpage}
\newpage

\tableofcontents

% \afterpage{\blankpage}
% \newpage
% \listoffigures

\addcontentsline{toc}{chapter}{Summary}
\chapter*{Summary}
\subsection*{CHAPTER 1: INTRODUCTION}
Over the past 150 year supramolecular chemistry has grown at an exponential rate, with the field expanding to embrace a wide range of applications, in what can be described as a new chemical space. This chapter aims to provide a brief historical account of the most notable development in supramolecular chemistry, focusing on MOFs and 3D periodic structures as well as their implementation in gas adsorption and electrochemistry applications.
\newline\subsection*{CHAPTER 2: RESULTS AND DISCUSSION}
The construction of 3D periodic solid structure oriented towards specific gas adsorption remain a challenging synthetic problem for chemists, several problems related to their development and use must be accounted. In this chapter will be reported the advancement in the synthesis of a pyrazole based ligand and the utilization of a 1,3-diketones ligand in electrochemistry oriented application.
\newline\subsection*{CHAPTER 3: EXPERIMENTAL SECTION}
Several synthetic procedures and characterization methods have been applied, both in the synthesis of pyrazole based ligand and 1,3-diketones MOF electrochemistry exploration, in order to achieve the results discussed earlier. They will be analyzed in great detail in this chapter.
\newline\subsection*{CHAPTER 4: TOOL DEVELOPMENT}
During the time spent working on these wide range topics the need of small and reliable analytical tools raised. Being able to analyze and draw a comparison between obtained results in a fast manner has been very important. Across this chapter will be presented a tiny python implementation for fast data analysis and graphs creation based on matplotlib.


\afterpage{\blankpage}
\newpage

% back to standard number
\pagenumbering{arabic}

% First chapter
\subfile{Chapters/Introduction.tex}

% Second chapter
\subfile{Chapters/ResultAndDiscussion.tex}

% Third chapter
\subfile{Chapters/ExperimentalSection.tex}

% Fourth chapter
\subfile{Chapters/ToolDevelopment.tex}

\afterpage{\blankpage}
\cleardoublepage

\printbibliography[heading=bibintoc]

\subfile{Chapters/Appendix.tex}

\end{document}
