\documentclass[../Master.tex]{subfiles}
\begin{document}
\chapter{Introduction}
\section{Supramolecolar Chemistry, why?}
\label{sec:supramolecular-chemistry}
Supramolecular chemistry can be classified as the branch of chemistry concerned about the interplay between designed molecular assemblies and intermolecular bonds, or more colloquially referred to as "chemistry beyond the molecule". \\
The discipline focuses on the design and synthesis of molecular architectures by relying on the complementary recognition, and subsequent assembly, of well-defined subunits. The products of complementary synthesis, the so-called "supermolecules" are sustained by noncovalent interactions such as hydrogen bonding, halogen bonding, coordination forces and \(\pi-\pi\). \\
The emergence of supramolecular chemistry has directly influenced how efficiently chemists can design and synthesize desired frameworks. The development and application of the bottom up approach is widely successful, owing to the noncovalent forces that dictate structural and morphological properties, while producing structures that were previously inaccessible.

\subsection{Host-Guest Chemistry}

In supramolecular chemistry, host–guest chemistry describes complexes that are composed of two or more molecules or ions that are held together in unique structural relationships by forces other than those of full covalent bonds. Host–guest chemistry encompasses the idea of molecular recognition and interactions through non-covalent bonding. Non-covalent bonding is critical in maintaining the 3D structure of large molecules, such as proteins and is involved in many biological processes in which large molecules bind specifically but transiently to one another.\\
Host-guest interaction has raised dramatical attention since it was discovered. It is an important field, because many biological processes require the host-guest interaction, and it can be useful in some material designs.

\subsection{Metal-Organic Frameworks}

Metal-Organic Frameworks represent an exciting and rapidly growing area of research within the field of supramolecular chemistry.\\
MOFs' chemistry is a specific type of supramolecular chemistry that involve the coordination of metal ions with organic ligands to form highly porous and crystalline materials with a unique structure. \\
The metal ions act as nodes that are connected by the organic ligands to create a three-dimensional framework. The resulting structure has a large internal surface area and can adsorb gases and other molecules with high efficiency, making MOFs useful in a variety of applications, such as gas storage, catalysis, and drug delivery.

\subsection{Carboxylate–Based Metal–Organic Frameworks}

\end{document}
%%% Local Variables:
%%% mode: latex
%%% TeX-master: "../Master"
%%% End:
